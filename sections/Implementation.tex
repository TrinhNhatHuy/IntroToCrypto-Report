\section{Implementation}


\subsection{Core Functions}

\begin{table}[H]
    \centering
    \renewcommand{\arraystretch}{1.3}
    \begin{tabularx}{\textwidth}{|l|X|}
        \hline
        \textbf{Function}                            & \textbf{Purpose}                                  \\
        \hline
        \texttt{gcd(a, b)}                           & Euclidean algorithm for greatest common divisor   \\
        \hline
        \texttt{extended\_gcd(a, b)}                 & Extended Euclidean algorithm, returns $(g, x, y)$ \\
        \hline
        \texttt{modinv(a, m)}                        & Modular inverse $a^{-1} \bmod m$ via extended GCD \\
        \hline
        \texttt{is\_prime(n, k)}                     & Miller-Rabin primality test with $k$ rounds       \\
        \hline
        \texttt{generate\_keypair(p, q)}             & Computes $(e, n)$ and $(d, n)$ from primes $p, q$ \\
        \hline
        \texttt{hash\_with\_all(msg)}                & Hashes message with MD5, SHA-1, SHA-256, SHA-512  \\
        \hline
        \texttt{sign\_all(msg, priv, hashes)}        & Signs hash values: $S = H^d \bmod n$              \\
        \hline
        \texttt{verify\_all(msg, sigs, pub, hashes)} & Verifies: checks $S^e \bmod n \stackrel{?}{=} H$  \\
        \hline
    \end{tabularx}
    \caption{Summary of implemented functions.}
    \label{tab:functions}
\end{table}

\subsection{Execution Flow}

The program runs interactively through five steps:

\begin{enumerate}
    \item \textbf{Input:} User provides two distinct primes $p$, $q$ and a message string.
    \item \textbf{Hash Comparison:} The message is hashed with all four algorithms. Output displays the digest length (128–512 bits) and security rating of each.
    \item \textbf{Key Generation:} Computes $n$, $\phi(n)$, selects $e = 65537$, and derives $d$. Displays both keys and the verification $(e \times d) \bmod \phi(n) = 1$.
    \item \textbf{Signing:} Generates four signatures (one per hash algorithm) using $S = H^d \bmod n$.
    \item \textbf{Verification \& Tampering:} First verifies the original message (all algorithms return VALID). Then appends text to simulate tampering—all algorithms detect the modification (CAUGHT).
\end{enumerate}

\subsection{Output Screenshots}
% ── Chụp screenshots khi chạy chương trình rồi thêm vào đây ──
% Ví dụ:
 \begin{figure}[H]
   \centering
   \includegraphics[width=0.85\textwidth]{images/image.png}
   \caption{Final Output after implementation all of the steps}
 \end{figure}

