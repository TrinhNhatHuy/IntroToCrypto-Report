\section{Security Analysis}

\subsection{Why RSA Digital Signatures Are Secure}

The security of RSA rests on the \textbf{Integer Factorization Problem}: given a large composite number $n = p \times q$, it is computationally infeasible to recover $p$ and $q$. Without knowing $p$ and $q$, an attacker cannot compute $\phi(n)$, and therefore cannot derive the private exponent $d$ from the public exponent $e$.

For our demonstration with small primes (e.g., $p = 104729$, $q = 224737$), the key size is approximately 34 bits—trivially factorable. In practice, RSA keys of 2048 bits or larger are used, where factorization would require billions of years with current technology.

\subsection{Hash Algorithm Comparison}

The choice of hash function directly impacts the security of the signature scheme:

\begin{table}[H]
\centering
\begin{tabular}{|l|c|c|l|}
\hline
\textbf{Algorithm} & \textbf{Output (bits)} & \textbf{Status} & \textbf{Note} \\
\hline
MD5     & 128 & Broken   & Collision attacks demonstrated (2004) \\
SHA-1   & 160 & Weak     & First practical collision found (2017) \\
SHA-256 & 256 & Secure   & Industry standard, recommended \\
SHA-512 & 512 & Secure   & Strongest output, higher margin \\
\hline
\end{tabular}
\caption{Security status of hash algorithms used in the implementation.}
\label{tab:hash_security}
\end{table}

\subsection{Tampering Detection}

The program demonstrates that modifying even a single character of the signed message causes \textbf{all four hash algorithms} to produce completely different digests—known as the \textbf{avalanche effect}. Since the signature was computed on the original hash, the recovered value $S^e \bmod n$ no longer matches the new hash, and verification fails immediately. This guarantees \textbf{data integrity}.

\subsection{Limitations of Textbook RSA}

This implementation uses \textit{textbook RSA} for educational clarity. In production systems, padding schemes such as \textbf{RSA-PSS} (Probabilistic Signature Scheme) are mandatory to prevent:
\begin{itemize}
    \item \textbf{Existential forgery:} crafting valid signatures without the private key.
    \item \textbf{Deterministic weakness:} identical messages always produce identical signatures, leaking information.
\end{itemize}
Standards like PKCS\#1 v2.1 specify RSA-PSS as the recommended signature scheme.
