\section{Mathematical Foundation}

\subsection{Key Generation}
The RSA key pair consists of a public key $(e, n)$ and a private key $(d, n)$, generated as follows:

\textbf{Step 1.} Select two distinct large primes $p$ and $q$, verified using the \textbf{Miller-Rabin} primality test.

\textbf{Step 2.} Compute the modulus: $n = p \times q$.

\textbf{Step 3.} Compute Euler's totient: $\phi(n) = (p-1)(q-1)$.

\textbf{Step 4.} Choose public exponent $e$ such that $1 < e < \phi(n)$ and $\gcd(e, \phi(n)) = 1$. The standard choice is $e = 65537$.

\textbf{Step 5.} Compute private exponent $d = e^{-1} \bmod \phi(n)$ using the Extended Euclidean Algorithm:
$$ e \times d \equiv 1 \pmod{\phi(n)} $$

\subsection{Signing and Verification}
Let $H$ be the integer representation of the hash of the message, reduced modulo $n$.

\textbf{Signing} (private key): The signer computes the signature $S$:
$$ S \equiv H^d \pmod{n} $$

\textbf{Verification} (public key): The verifier recovers the hash:
$$ H' \equiv S^e \pmod{n} $$

By Euler's theorem, since $e \cdot d \equiv 1 \pmod{\phi(n)}$:
$$ S^e \equiv (H^d)^e \equiv H^{ed} \equiv H \pmod{n} $$
If $H' = H$, the signature is valid—proving both \textbf{authenticity} and \textbf{data integrity}.