\section{Introduction}

\subsection{Background and Motivation}
In the era of modern digital communication, data is constantly transmitted over public and inherently insecure networks. This reality makes sensitive information highly vulnerable to various cyber threats, including interception, unauthorized tampering, and forgery. Cryptography provides the essential tools to secure this data, not only by hiding its content through encryption but also by ensuring its authenticity. Digital signatures play a critical role in this security ecosystem, acting as the mathematical equivalent of handwritten signatures or stamped seals to establish trust in digital transactions.

\subsection{The Role of Digital Signatures}
The primary purpose of a digital signature mechanism is to provide two fundamental security guarantees: data integrity and authentication. Data integrity ensures that a message or document has not been maliciously or accidentally altered during transit. Authentication, coupled with the concept of non-repudiation, mathematically proves the identity of the sender, preventing them from falsely denying their involvement in the communication. The RSA (Rivest–Shamir–Adleman) algorithm, an asymmetric cryptographic system, combined with secure hash functions, serves as one of the most robust and widely adopted frameworks for achieving these security objectives.

\subsection{Objectives of the Report}
This midterm project aims to explore both the theoretical foundations and the practical applications of RSA digital signatures. Specifically, this report will detail the underlying mathematical principles of the RSA algorithm, including the key generation process and modular arithmetic operations. Furthermore, it will present a practical implementation written in Python to demonstrate how a message is hashed, signed using a private key, and subsequently verified using a corresponding public key. Finally, the report will address the limitations of textbook RSA implementations and highlight the necessity of secure practices in real-world scenarios.