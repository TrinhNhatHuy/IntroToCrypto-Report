\section{Introduction}

In modern digital communication, data is constantly transmitted over public, insecure networks, making it vulnerable to interception, tampering, and forgery. Cryptography provides the fundamental tools to address these threats—not only through encryption, but also through mechanisms that guarantee \textbf{authenticity} and \textbf{integrity} of information.

A \textbf{digital signature} is the mathematical equivalent of a handwritten signature: it binds the identity of a signer to the content of a message, ensuring that any modification after signing is immediately detectable. The RSA algorithm (Rivest–Shamir–Adleman), one of the most widely adopted asymmetric cryptosystems, serves as the foundation for this project.

This report explores both the theory and practice of RSA digital signatures. It covers the mathematical foundations (key generation, modular arithmetic), presents the system architecture and a working Python implementation, analyzes security properties, and evaluates performance across different hash algorithms.